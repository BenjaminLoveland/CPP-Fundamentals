\documentclass[11pt]{article}

    \usepackage[breakable]{tcolorbox}
    \usepackage{parskip} % Stop auto-indenting (to mimic markdown behaviour)
    
    \usepackage{iftex}
    \ifPDFTeX
    	\usepackage[T1]{fontenc}
    	\usepackage{mathpazo}
    \else
    	\usepackage{fontspec}
    \fi

    % Basic figure setup, for now with no caption control since it's done
    % automatically by Pandoc (which extracts ![](path) syntax from Markdown).
    \usepackage{graphicx}
    % Maintain compatibility with old templates. Remove in nbconvert 6.0
    \let\Oldincludegraphics\includegraphics
    % Ensure that by default, figures have no caption (until we provide a
    % proper Figure object with a Caption API and a way to capture that
    % in the conversion process - todo).
    \usepackage{caption}
    \DeclareCaptionFormat{nocaption}{}
    \captionsetup{format=nocaption,aboveskip=0pt,belowskip=0pt}

    \usepackage{float}
    \floatplacement{figure}{H} % forces figures to be placed at the correct location
    \usepackage{xcolor} % Allow colors to be defined
    \usepackage{enumerate} % Needed for markdown enumerations to work
    \usepackage{geometry} % Used to adjust the document margins
    \usepackage{amsmath} % Equations
    \usepackage{amssymb} % Equations
    \usepackage{textcomp} % defines textquotesingle
    % Hack from http://tex.stackexchange.com/a/47451/13684:
    \AtBeginDocument{%
        \def\PYZsq{\textquotesingle}% Upright quotes in Pygmentized code
    }
    \usepackage{upquote} % Upright quotes for verbatim code
    \usepackage{eurosym} % defines \euro
    \usepackage[mathletters]{ucs} % Extended unicode (utf-8) support
    \usepackage{fancyvrb} % verbatim replacement that allows latex
    \usepackage{grffile} % extends the file name processing of package graphics 
                         % to support a larger range
    \makeatletter % fix for old versions of grffile with XeLaTeX
    \@ifpackagelater{grffile}{2019/11/01}
    {
      % Do nothing on new versions
    }
    {
      \def\Gread@@xetex#1{%
        \IfFileExists{"\Gin@base".bb}%
        {\Gread@eps{\Gin@base.bb}}%
        {\Gread@@xetex@aux#1}%
      }
    }
    \makeatother
    \usepackage[Export]{adjustbox} % Used to constrain images to a maximum size
    \adjustboxset{max size={0.9\linewidth}{0.9\paperheight}}

    % The hyperref package gives us a pdf with properly built
    % internal navigation ('pdf bookmarks' for the table of contents,
    % internal cross-reference links, web links for URLs, etc.)
    \usepackage{hyperref}
    % The default LaTeX title has an obnoxious amount of whitespace. By default,
    % titling removes some of it. It also provides customization options.
    \usepackage{titling}
    \usepackage{longtable} % longtable support required by pandoc >1.10
    \usepackage{booktabs}  % table support for pandoc > 1.12.2
    \usepackage[inline]{enumitem} % IRkernel/repr support (it uses the enumerate* environment)
    \usepackage[normalem]{ulem} % ulem is needed to support strikethroughs (\sout)
                                % normalem makes italics be italics, not underlines
    \usepackage{mathrsfs}
    

    
    % Colors for the hyperref package
    \definecolor{urlcolor}{rgb}{0,.145,.698}
    \definecolor{linkcolor}{rgb}{.71,0.21,0.01}
    \definecolor{citecolor}{rgb}{.12,.54,.11}

    % ANSI colors
    \definecolor{ansi-black}{HTML}{3E424D}
    \definecolor{ansi-black-intense}{HTML}{282C36}
    \definecolor{ansi-red}{HTML}{E75C58}
    \definecolor{ansi-red-intense}{HTML}{B22B31}
    \definecolor{ansi-green}{HTML}{00A250}
    \definecolor{ansi-green-intense}{HTML}{007427}
    \definecolor{ansi-yellow}{HTML}{DDB62B}
    \definecolor{ansi-yellow-intense}{HTML}{B27D12}
    \definecolor{ansi-blue}{HTML}{208FFB}
    \definecolor{ansi-blue-intense}{HTML}{0065CA}
    \definecolor{ansi-magenta}{HTML}{D160C4}
    \definecolor{ansi-magenta-intense}{HTML}{A03196}
    \definecolor{ansi-cyan}{HTML}{60C6C8}
    \definecolor{ansi-cyan-intense}{HTML}{258F8F}
    \definecolor{ansi-white}{HTML}{C5C1B4}
    \definecolor{ansi-white-intense}{HTML}{A1A6B2}
    \definecolor{ansi-default-inverse-fg}{HTML}{FFFFFF}
    \definecolor{ansi-default-inverse-bg}{HTML}{000000}

    % common color for the border for error outputs.
    \definecolor{outerrorbackground}{HTML}{FFDFDF}

    % commands and environments needed by pandoc snippets
    % extracted from the output of `pandoc -s`
    \providecommand{\tightlist}{%
      \setlength{\itemsep}{0pt}\setlength{\parskip}{0pt}}
    \DefineVerbatimEnvironment{Highlighting}{Verbatim}{commandchars=\\\{\}}
    % Add ',fontsize=\small' for more characters per line
    \newenvironment{Shaded}{}{}
    \newcommand{\KeywordTok}[1]{\textcolor[rgb]{0.00,0.44,0.13}{\textbf{{#1}}}}
    \newcommand{\DataTypeTok}[1]{\textcolor[rgb]{0.56,0.13,0.00}{{#1}}}
    \newcommand{\DecValTok}[1]{\textcolor[rgb]{0.25,0.63,0.44}{{#1}}}
    \newcommand{\BaseNTok}[1]{\textcolor[rgb]{0.25,0.63,0.44}{{#1}}}
    \newcommand{\FloatTok}[1]{\textcolor[rgb]{0.25,0.63,0.44}{{#1}}}
    \newcommand{\CharTok}[1]{\textcolor[rgb]{0.25,0.44,0.63}{{#1}}}
    \newcommand{\StringTok}[1]{\textcolor[rgb]{0.25,0.44,0.63}{{#1}}}
    \newcommand{\CommentTok}[1]{\textcolor[rgb]{0.38,0.63,0.69}{\textit{{#1}}}}
    \newcommand{\OtherTok}[1]{\textcolor[rgb]{0.00,0.44,0.13}{{#1}}}
    \newcommand{\AlertTok}[1]{\textcolor[rgb]{1.00,0.00,0.00}{\textbf{{#1}}}}
    \newcommand{\FunctionTok}[1]{\textcolor[rgb]{0.02,0.16,0.49}{{#1}}}
    \newcommand{\RegionMarkerTok}[1]{{#1}}
    \newcommand{\ErrorTok}[1]{\textcolor[rgb]{1.00,0.00,0.00}{\textbf{{#1}}}}
    \newcommand{\NormalTok}[1]{{#1}}
    
    % Additional commands for more recent versions of Pandoc
    \newcommand{\ConstantTok}[1]{\textcolor[rgb]{0.53,0.00,0.00}{{#1}}}
    \newcommand{\SpecialCharTok}[1]{\textcolor[rgb]{0.25,0.44,0.63}{{#1}}}
    \newcommand{\VerbatimStringTok}[1]{\textcolor[rgb]{0.25,0.44,0.63}{{#1}}}
    \newcommand{\SpecialStringTok}[1]{\textcolor[rgb]{0.73,0.40,0.53}{{#1}}}
    \newcommand{\ImportTok}[1]{{#1}}
    \newcommand{\DocumentationTok}[1]{\textcolor[rgb]{0.73,0.13,0.13}{\textit{{#1}}}}
    \newcommand{\AnnotationTok}[1]{\textcolor[rgb]{0.38,0.63,0.69}{\textbf{\textit{{#1}}}}}
    \newcommand{\CommentVarTok}[1]{\textcolor[rgb]{0.38,0.63,0.69}{\textbf{\textit{{#1}}}}}
    \newcommand{\VariableTok}[1]{\textcolor[rgb]{0.10,0.09,0.49}{{#1}}}
    \newcommand{\ControlFlowTok}[1]{\textcolor[rgb]{0.00,0.44,0.13}{\textbf{{#1}}}}
    \newcommand{\OperatorTok}[1]{\textcolor[rgb]{0.40,0.40,0.40}{{#1}}}
    \newcommand{\BuiltInTok}[1]{{#1}}
    \newcommand{\ExtensionTok}[1]{{#1}}
    \newcommand{\PreprocessorTok}[1]{\textcolor[rgb]{0.74,0.48,0.00}{{#1}}}
    \newcommand{\AttributeTok}[1]{\textcolor[rgb]{0.49,0.56,0.16}{{#1}}}
    \newcommand{\InformationTok}[1]{\textcolor[rgb]{0.38,0.63,0.69}{\textbf{\textit{{#1}}}}}
    \newcommand{\WarningTok}[1]{\textcolor[rgb]{0.38,0.63,0.69}{\textbf{\textit{{#1}}}}}
    
    
    % Define a nice break command that doesn't care if a line doesn't already
    % exist.
    \def\br{\hspace*{\fill} \\* }
    % Math Jax compatibility definitions
    \def\gt{>}
    \def\lt{<}
    \let\Oldtex\TeX
    \let\Oldlatex\LaTeX
    \renewcommand{\TeX}{\textrm{\Oldtex}}
    \renewcommand{\LaTeX}{\textrm{\Oldlatex}}
    % Document parameters
    % Document title
    \title{FileIO}
    
    
    
    
    
% Pygments definitions
\makeatletter
\def\PY@reset{\let\PY@it=\relax \let\PY@bf=\relax%
    \let\PY@ul=\relax \let\PY@tc=\relax%
    \let\PY@bc=\relax \let\PY@ff=\relax}
\def\PY@tok#1{\csname PY@tok@#1\endcsname}
\def\PY@toks#1+{\ifx\relax#1\empty\else%
    \PY@tok{#1}\expandafter\PY@toks\fi}
\def\PY@do#1{\PY@bc{\PY@tc{\PY@ul{%
    \PY@it{\PY@bf{\PY@ff{#1}}}}}}}
\def\PY#1#2{\PY@reset\PY@toks#1+\relax+\PY@do{#2}}

\@namedef{PY@tok@w}{\def\PY@tc##1{\textcolor[rgb]{0.73,0.73,0.73}{##1}}}
\@namedef{PY@tok@c}{\let\PY@it=\textit\def\PY@tc##1{\textcolor[rgb]{0.25,0.50,0.50}{##1}}}
\@namedef{PY@tok@cp}{\def\PY@tc##1{\textcolor[rgb]{0.74,0.48,0.00}{##1}}}
\@namedef{PY@tok@k}{\let\PY@bf=\textbf\def\PY@tc##1{\textcolor[rgb]{0.00,0.50,0.00}{##1}}}
\@namedef{PY@tok@kp}{\def\PY@tc##1{\textcolor[rgb]{0.00,0.50,0.00}{##1}}}
\@namedef{PY@tok@kt}{\def\PY@tc##1{\textcolor[rgb]{0.69,0.00,0.25}{##1}}}
\@namedef{PY@tok@o}{\def\PY@tc##1{\textcolor[rgb]{0.40,0.40,0.40}{##1}}}
\@namedef{PY@tok@ow}{\let\PY@bf=\textbf\def\PY@tc##1{\textcolor[rgb]{0.67,0.13,1.00}{##1}}}
\@namedef{PY@tok@nb}{\def\PY@tc##1{\textcolor[rgb]{0.00,0.50,0.00}{##1}}}
\@namedef{PY@tok@nf}{\def\PY@tc##1{\textcolor[rgb]{0.00,0.00,1.00}{##1}}}
\@namedef{PY@tok@nc}{\let\PY@bf=\textbf\def\PY@tc##1{\textcolor[rgb]{0.00,0.00,1.00}{##1}}}
\@namedef{PY@tok@nn}{\let\PY@bf=\textbf\def\PY@tc##1{\textcolor[rgb]{0.00,0.00,1.00}{##1}}}
\@namedef{PY@tok@ne}{\let\PY@bf=\textbf\def\PY@tc##1{\textcolor[rgb]{0.82,0.25,0.23}{##1}}}
\@namedef{PY@tok@nv}{\def\PY@tc##1{\textcolor[rgb]{0.10,0.09,0.49}{##1}}}
\@namedef{PY@tok@no}{\def\PY@tc##1{\textcolor[rgb]{0.53,0.00,0.00}{##1}}}
\@namedef{PY@tok@nl}{\def\PY@tc##1{\textcolor[rgb]{0.63,0.63,0.00}{##1}}}
\@namedef{PY@tok@ni}{\let\PY@bf=\textbf\def\PY@tc##1{\textcolor[rgb]{0.60,0.60,0.60}{##1}}}
\@namedef{PY@tok@na}{\def\PY@tc##1{\textcolor[rgb]{0.49,0.56,0.16}{##1}}}
\@namedef{PY@tok@nt}{\let\PY@bf=\textbf\def\PY@tc##1{\textcolor[rgb]{0.00,0.50,0.00}{##1}}}
\@namedef{PY@tok@nd}{\def\PY@tc##1{\textcolor[rgb]{0.67,0.13,1.00}{##1}}}
\@namedef{PY@tok@s}{\def\PY@tc##1{\textcolor[rgb]{0.73,0.13,0.13}{##1}}}
\@namedef{PY@tok@sd}{\let\PY@it=\textit\def\PY@tc##1{\textcolor[rgb]{0.73,0.13,0.13}{##1}}}
\@namedef{PY@tok@si}{\let\PY@bf=\textbf\def\PY@tc##1{\textcolor[rgb]{0.73,0.40,0.53}{##1}}}
\@namedef{PY@tok@se}{\let\PY@bf=\textbf\def\PY@tc##1{\textcolor[rgb]{0.73,0.40,0.13}{##1}}}
\@namedef{PY@tok@sr}{\def\PY@tc##1{\textcolor[rgb]{0.73,0.40,0.53}{##1}}}
\@namedef{PY@tok@ss}{\def\PY@tc##1{\textcolor[rgb]{0.10,0.09,0.49}{##1}}}
\@namedef{PY@tok@sx}{\def\PY@tc##1{\textcolor[rgb]{0.00,0.50,0.00}{##1}}}
\@namedef{PY@tok@m}{\def\PY@tc##1{\textcolor[rgb]{0.40,0.40,0.40}{##1}}}
\@namedef{PY@tok@gh}{\let\PY@bf=\textbf\def\PY@tc##1{\textcolor[rgb]{0.00,0.00,0.50}{##1}}}
\@namedef{PY@tok@gu}{\let\PY@bf=\textbf\def\PY@tc##1{\textcolor[rgb]{0.50,0.00,0.50}{##1}}}
\@namedef{PY@tok@gd}{\def\PY@tc##1{\textcolor[rgb]{0.63,0.00,0.00}{##1}}}
\@namedef{PY@tok@gi}{\def\PY@tc##1{\textcolor[rgb]{0.00,0.63,0.00}{##1}}}
\@namedef{PY@tok@gr}{\def\PY@tc##1{\textcolor[rgb]{1.00,0.00,0.00}{##1}}}
\@namedef{PY@tok@ge}{\let\PY@it=\textit}
\@namedef{PY@tok@gs}{\let\PY@bf=\textbf}
\@namedef{PY@tok@gp}{\let\PY@bf=\textbf\def\PY@tc##1{\textcolor[rgb]{0.00,0.00,0.50}{##1}}}
\@namedef{PY@tok@go}{\def\PY@tc##1{\textcolor[rgb]{0.53,0.53,0.53}{##1}}}
\@namedef{PY@tok@gt}{\def\PY@tc##1{\textcolor[rgb]{0.00,0.27,0.87}{##1}}}
\@namedef{PY@tok@err}{\def\PY@bc##1{{\setlength{\fboxsep}{\string -\fboxrule}\fcolorbox[rgb]{1.00,0.00,0.00}{1,1,1}{\strut ##1}}}}
\@namedef{PY@tok@kc}{\let\PY@bf=\textbf\def\PY@tc##1{\textcolor[rgb]{0.00,0.50,0.00}{##1}}}
\@namedef{PY@tok@kd}{\let\PY@bf=\textbf\def\PY@tc##1{\textcolor[rgb]{0.00,0.50,0.00}{##1}}}
\@namedef{PY@tok@kn}{\let\PY@bf=\textbf\def\PY@tc##1{\textcolor[rgb]{0.00,0.50,0.00}{##1}}}
\@namedef{PY@tok@kr}{\let\PY@bf=\textbf\def\PY@tc##1{\textcolor[rgb]{0.00,0.50,0.00}{##1}}}
\@namedef{PY@tok@bp}{\def\PY@tc##1{\textcolor[rgb]{0.00,0.50,0.00}{##1}}}
\@namedef{PY@tok@fm}{\def\PY@tc##1{\textcolor[rgb]{0.00,0.00,1.00}{##1}}}
\@namedef{PY@tok@vc}{\def\PY@tc##1{\textcolor[rgb]{0.10,0.09,0.49}{##1}}}
\@namedef{PY@tok@vg}{\def\PY@tc##1{\textcolor[rgb]{0.10,0.09,0.49}{##1}}}
\@namedef{PY@tok@vi}{\def\PY@tc##1{\textcolor[rgb]{0.10,0.09,0.49}{##1}}}
\@namedef{PY@tok@vm}{\def\PY@tc##1{\textcolor[rgb]{0.10,0.09,0.49}{##1}}}
\@namedef{PY@tok@sa}{\def\PY@tc##1{\textcolor[rgb]{0.73,0.13,0.13}{##1}}}
\@namedef{PY@tok@sb}{\def\PY@tc##1{\textcolor[rgb]{0.73,0.13,0.13}{##1}}}
\@namedef{PY@tok@sc}{\def\PY@tc##1{\textcolor[rgb]{0.73,0.13,0.13}{##1}}}
\@namedef{PY@tok@dl}{\def\PY@tc##1{\textcolor[rgb]{0.73,0.13,0.13}{##1}}}
\@namedef{PY@tok@s2}{\def\PY@tc##1{\textcolor[rgb]{0.73,0.13,0.13}{##1}}}
\@namedef{PY@tok@sh}{\def\PY@tc##1{\textcolor[rgb]{0.73,0.13,0.13}{##1}}}
\@namedef{PY@tok@s1}{\def\PY@tc##1{\textcolor[rgb]{0.73,0.13,0.13}{##1}}}
\@namedef{PY@tok@mb}{\def\PY@tc##1{\textcolor[rgb]{0.40,0.40,0.40}{##1}}}
\@namedef{PY@tok@mf}{\def\PY@tc##1{\textcolor[rgb]{0.40,0.40,0.40}{##1}}}
\@namedef{PY@tok@mh}{\def\PY@tc##1{\textcolor[rgb]{0.40,0.40,0.40}{##1}}}
\@namedef{PY@tok@mi}{\def\PY@tc##1{\textcolor[rgb]{0.40,0.40,0.40}{##1}}}
\@namedef{PY@tok@il}{\def\PY@tc##1{\textcolor[rgb]{0.40,0.40,0.40}{##1}}}
\@namedef{PY@tok@mo}{\def\PY@tc##1{\textcolor[rgb]{0.40,0.40,0.40}{##1}}}
\@namedef{PY@tok@ch}{\let\PY@it=\textit\def\PY@tc##1{\textcolor[rgb]{0.25,0.50,0.50}{##1}}}
\@namedef{PY@tok@cm}{\let\PY@it=\textit\def\PY@tc##1{\textcolor[rgb]{0.25,0.50,0.50}{##1}}}
\@namedef{PY@tok@cpf}{\let\PY@it=\textit\def\PY@tc##1{\textcolor[rgb]{0.25,0.50,0.50}{##1}}}
\@namedef{PY@tok@c1}{\let\PY@it=\textit\def\PY@tc##1{\textcolor[rgb]{0.25,0.50,0.50}{##1}}}
\@namedef{PY@tok@cs}{\let\PY@it=\textit\def\PY@tc##1{\textcolor[rgb]{0.25,0.50,0.50}{##1}}}

\def\PYZbs{\char`\\}
\def\PYZus{\char`\_}
\def\PYZob{\char`\{}
\def\PYZcb{\char`\}}
\def\PYZca{\char`\^}
\def\PYZam{\char`\&}
\def\PYZlt{\char`\<}
\def\PYZgt{\char`\>}
\def\PYZsh{\char`\#}
\def\PYZpc{\char`\%}
\def\PYZdl{\char`\$}
\def\PYZhy{\char`\-}
\def\PYZsq{\char`\'}
\def\PYZdq{\char`\"}
\def\PYZti{\char`\~}
% for compatibility with earlier versions
\def\PYZat{@}
\def\PYZlb{[}
\def\PYZrb{]}
\makeatother


    % For linebreaks inside Verbatim environment from package fancyvrb. 
    \makeatletter
        \newbox\Wrappedcontinuationbox 
        \newbox\Wrappedvisiblespacebox 
        \newcommand*\Wrappedvisiblespace {\textcolor{red}{\textvisiblespace}} 
        \newcommand*\Wrappedcontinuationsymbol {\textcolor{red}{\llap{\tiny$\m@th\hookrightarrow$}}} 
        \newcommand*\Wrappedcontinuationindent {3ex } 
        \newcommand*\Wrappedafterbreak {\kern\Wrappedcontinuationindent\copy\Wrappedcontinuationbox} 
        % Take advantage of the already applied Pygments mark-up to insert 
        % potential linebreaks for TeX processing. 
        %        {, <, #, %, $, ' and ": go to next line. 
        %        _, }, ^, &, >, - and ~: stay at end of broken line. 
        % Use of \textquotesingle for straight quote. 
        \newcommand*\Wrappedbreaksatspecials {% 
            \def\PYGZus{\discretionary{\char`\_}{\Wrappedafterbreak}{\char`\_}}% 
            \def\PYGZob{\discretionary{}{\Wrappedafterbreak\char`\{}{\char`\{}}% 
            \def\PYGZcb{\discretionary{\char`\}}{\Wrappedafterbreak}{\char`\}}}% 
            \def\PYGZca{\discretionary{\char`\^}{\Wrappedafterbreak}{\char`\^}}% 
            \def\PYGZam{\discretionary{\char`\&}{\Wrappedafterbreak}{\char`\&}}% 
            \def\PYGZlt{\discretionary{}{\Wrappedafterbreak\char`\<}{\char`\<}}% 
            \def\PYGZgt{\discretionary{\char`\>}{\Wrappedafterbreak}{\char`\>}}% 
            \def\PYGZsh{\discretionary{}{\Wrappedafterbreak\char`\#}{\char`\#}}% 
            \def\PYGZpc{\discretionary{}{\Wrappedafterbreak\char`\%}{\char`\%}}% 
            \def\PYGZdl{\discretionary{}{\Wrappedafterbreak\char`\$}{\char`\$}}% 
            \def\PYGZhy{\discretionary{\char`\-}{\Wrappedafterbreak}{\char`\-}}% 
            \def\PYGZsq{\discretionary{}{\Wrappedafterbreak\textquotesingle}{\textquotesingle}}% 
            \def\PYGZdq{\discretionary{}{\Wrappedafterbreak\char`\"}{\char`\"}}% 
            \def\PYGZti{\discretionary{\char`\~}{\Wrappedafterbreak}{\char`\~}}% 
        } 
        % Some characters . , ; ? ! / are not pygmentized. 
        % This macro makes them "active" and they will insert potential linebreaks 
        \newcommand*\Wrappedbreaksatpunct {% 
            \lccode`\~`\.\lowercase{\def~}{\discretionary{\hbox{\char`\.}}{\Wrappedafterbreak}{\hbox{\char`\.}}}% 
            \lccode`\~`\,\lowercase{\def~}{\discretionary{\hbox{\char`\,}}{\Wrappedafterbreak}{\hbox{\char`\,}}}% 
            \lccode`\~`\;\lowercase{\def~}{\discretionary{\hbox{\char`\;}}{\Wrappedafterbreak}{\hbox{\char`\;}}}% 
            \lccode`\~`\:\lowercase{\def~}{\discretionary{\hbox{\char`\:}}{\Wrappedafterbreak}{\hbox{\char`\:}}}% 
            \lccode`\~`\?\lowercase{\def~}{\discretionary{\hbox{\char`\?}}{\Wrappedafterbreak}{\hbox{\char`\?}}}% 
            \lccode`\~`\!\lowercase{\def~}{\discretionary{\hbox{\char`\!}}{\Wrappedafterbreak}{\hbox{\char`\!}}}% 
            \lccode`\~`\/\lowercase{\def~}{\discretionary{\hbox{\char`\/}}{\Wrappedafterbreak}{\hbox{\char`\/}}}% 
            \catcode`\.\active
            \catcode`\,\active 
            \catcode`\;\active
            \catcode`\:\active
            \catcode`\?\active
            \catcode`\!\active
            \catcode`\/\active 
            \lccode`\~`\~ 	
        }
    \makeatother

    \let\OriginalVerbatim=\Verbatim
    \makeatletter
    \renewcommand{\Verbatim}[1][1]{%
        %\parskip\z@skip
        \sbox\Wrappedcontinuationbox {\Wrappedcontinuationsymbol}%
        \sbox\Wrappedvisiblespacebox {\FV@SetupFont\Wrappedvisiblespace}%
        \def\FancyVerbFormatLine ##1{\hsize\linewidth
            \vtop{\raggedright\hyphenpenalty\z@\exhyphenpenalty\z@
                \doublehyphendemerits\z@\finalhyphendemerits\z@
                \strut ##1\strut}%
        }%
        % If the linebreak is at a space, the latter will be displayed as visible
        % space at end of first line, and a continuation symbol starts next line.
        % Stretch/shrink are however usually zero for typewriter font.
        \def\FV@Space {%
            \nobreak\hskip\z@ plus\fontdimen3\font minus\fontdimen4\font
            \discretionary{\copy\Wrappedvisiblespacebox}{\Wrappedafterbreak}
            {\kern\fontdimen2\font}%
        }%
        
        % Allow breaks at special characters using \PYG... macros.
        \Wrappedbreaksatspecials
        % Breaks at punctuation characters . , ; ? ! and / need catcode=\active 	
        \OriginalVerbatim[#1,codes*=\Wrappedbreaksatpunct]%
    }
    \makeatother

    % Exact colors from NB
    \definecolor{incolor}{HTML}{303F9F}
    \definecolor{outcolor}{HTML}{D84315}
    \definecolor{cellborder}{HTML}{CFCFCF}
    \definecolor{cellbackground}{HTML}{F7F7F7}
    
    % prompt
    \makeatletter
    \newcommand{\boxspacing}{\kern\kvtcb@left@rule\kern\kvtcb@boxsep}
    \makeatother
    \newcommand{\prompt}[4]{
        {\ttfamily\llap{{\color{#2}[#3]:\hspace{3pt}#4}}\vspace{-\baselineskip}}
    }
    

    
    % Prevent overflowing lines due to hard-to-break entities
    \sloppy 
    % Setup hyperref package
    \hypersetup{
      breaklinks=true,  % so long urls are correctly broken across lines
      colorlinks=true,
      urlcolor=urlcolor,
      linkcolor=linkcolor,
      citecolor=citecolor,
      }
    % Slightly bigger margins than the latex defaults
    
    \geometry{verbose,tmargin=1in,bmargin=1in,lmargin=1in,rmargin=1in}
    
    

\begin{document}
    
    \maketitle
    
    

    
    \hypertarget{file-inputoutput-io}{%
\section{File Input/Output (IO)}\label{file-inputoutput-io}}

\hypertarget{topics}{%
\subsection{Topics}\label{topics}}

\begin{itemize}
\tightlist
\item
  input/output streams
\item
  file input stream
\item
  file output stream
\item
  reading unstructured and structured text files
\item
  formatting file output
\end{itemize}

    \hypertarget{streams}{%
\subsection{Streams}\label{streams}}

\begin{itemize}
\tightlist
\item
  a \textbf{stream} is an abstract object that represents the flow of
  data from a source like keyboard or a file to a destination like the
  screen or a file
\item
  we've learned about standard io streams in earlier chapters
\item
  iostream is used to read the data from standard input (keyboard)

  \begin{itemize}
  \tightlist
  \item
    data is then stored in computer memory to be manipulated to solve
    problems
  \item
    result is written to the standard output (monitor) from computer
    memory
  \end{itemize}
\item
  C++ uses various streams to read data from and write data to

  \begin{itemize}
  \tightlist
  \item
    stringstream is another stream that creates stream of strings
  \end{itemize}
\item
  often programs need to read data, process it and write the result back
  to secondary devices for permanent storage
\item
  file stream is used to read data from secondary storage (e.g., hard
  disk and flash drive) and write result and data back to it for
  permanent storage
\end{itemize}

    \hypertarget{file-stream}{%
\subsection{File stream}\label{file-stream}}

\begin{itemize}
\tightlist
\item
  we use \texttt{\textless{}fstream\textgreater{}} header to create
  input and output file streams
\end{itemize}

\hypertarget{file-input}{%
\subsection{File input}\label{file-input}}

\begin{itemize}
\tightlist
\item
  \textbf{ifstream} object is created to read data from file
\item
  it creates a stream that flows from the file into the program (memory)
\end{itemize}

\hypertarget{steps-for-file-input}{%
\subsubsection{Steps for file input}\label{steps-for-file-input}}

\begin{enumerate}
\def\labelenumi{\arabic{enumi}.}
\tightlist
\item
  open file to read data from

  \begin{itemize}
  \tightlist
  \item
    file must exist; run-time error otherwise
  \end{itemize}
\item
  read file contents
\item
  close the file
\end{enumerate}

\hypertarget{open-file}{%
\subsubsection{Open file}\label{open-file}}

\begin{itemize}
\tightlist
\item
  to open the file you need to create \texttt{ifstream} object
\item
  then open the file using the object
\item
  syntax to create \texttt{ifstream} object:
\end{itemize}

\begin{Shaded}
\begin{Highlighting}[]
\CommentTok{//1. create stream object without opening the file}
\NormalTok{ifstream objectName}\OperatorTok{;}
\CommentTok{//2. open a file with the objectName}
\NormalTok{objectName}\OperatorTok{.}\NormalTok{open}\OperatorTok{(}\StringTok{"fileName"}\OperatorTok{);}

\CommentTok{// OR 1. create object and open the given file}
\NormalTok{ifstream objectName}\OperatorTok{(}\StringTok{"file\_name"}\OperatorTok{);}
\end{Highlighting}
\end{Shaded}

\begin{itemize}
\item
  objectName is any identifier you want to use it for this particular
  ifstream
\item
  file name is passed as an argument; we'll learn how to read text files
\item
  file name must be present to read data from
\item
  let's open and read this sample text file called
  \url{demos/file_io/inputfile.txt}
\end{itemize}

    \begin{tcolorbox}[breakable, size=fbox, boxrule=1pt, pad at break*=1mm,colback=cellbackground, colframe=cellborder]
\prompt{In}{incolor}{1}{\boxspacing}
\begin{Verbatim}[commandchars=\\\{\}]
\PY{c+cp}{\PYZsh{}}\PY{c+cp}{include} \PY{c+cpf}{\PYZlt{}fstream\PYZgt{}}\PY{c+c1}{ // ifstream}
\PY{c+cp}{\PYZsh{}}\PY{c+cp}{include} \PY{c+cpf}{\PYZlt{}iostream\PYZgt{}}
\PY{c+cp}{\PYZsh{}}\PY{c+cp}{include} \PY{c+cpf}{\PYZlt{}string\PYZgt{}}

\PY{k}{using} \PY{k}{namespace} \PY{n+nn}{std}\PY{p}{;}
\end{Verbatim}
\end{tcolorbox}

    \begin{tcolorbox}[breakable, size=fbox, boxrule=1pt, pad at break*=1mm,colback=cellbackground, colframe=cellborder]
\prompt{In}{incolor}{2}{\boxspacing}
\begin{Verbatim}[commandchars=\\\{\}]
\PY{c+c1}{// declare ifstream object}
\PY{n}{ifstream} \PY{n}{fin}\PY{p}{;}
\PY{c+c1}{// i prefer fin as stream object name; rhymes with cin}
\end{Verbatim}
\end{tcolorbox}

    \begin{tcolorbox}[breakable, size=fbox, boxrule=1pt, pad at break*=1mm,colback=cellbackground, colframe=cellborder]
\prompt{In}{incolor}{3}{\boxspacing}
\begin{Verbatim}[commandchars=\\\{\}]
\PY{c+c1}{// open the file using open method}
\PY{n}{fin}\PY{p}{.}\PY{n}{open}\PY{p}{(}\PY{l+s}{\PYZdq{}}\PY{l+s}{./demos/file\PYZus{}io/inputfile.txt}\PY{l+s}{\PYZdq{}}\PY{p}{)}\PY{p}{;}
\end{Verbatim}
\end{tcolorbox}

    \begin{tcolorbox}[breakable, size=fbox, boxrule=1pt, pad at break*=1mm,colback=cellbackground, colframe=cellborder]
\prompt{In}{incolor}{4}{\boxspacing}
\begin{Verbatim}[commandchars=\\\{\}]
\PY{c+c1}{// declare stream object and open the given file}
\PY{n}{ifstream} \PY{n+nf}{fin1}\PY{p}{(}\PY{l+s}{\PYZdq{}}\PY{l+s}{./demos/file\PYZus{}io/inputfile.txt}\PY{l+s}{\PYZdq{}}\PY{p}{)}\PY{p}{;}
\end{Verbatim}
\end{tcolorbox}

    \hypertarget{read-data}{%
\subsubsection{Read data}\label{read-data}}

\begin{itemize}
\tightlist
\item
  once the ifstream object is created and file opened, reading data is
  similar to reading from iostream
\item
  we use \texttt{\textgreater{}\textgreater{}} input extraction operator
  and getline functions to read the data

  \begin{itemize}
  \tightlist
  \item
    similar to standard io
  \end{itemize}
\item
  syntax:
\end{itemize}

\begin{Shaded}
\begin{Highlighting}[]
\NormalTok{ifstreamObject }\OperatorTok{\textgreater{}\textgreater{}}\NormalTok{ variable1 }\OperatorTok{\textgreater{}\textgreater{}}\NormalTok{ variable2 }\OperatorTok{\textgreater{}\textgreater{}} \OperatorTok{...;}
\end{Highlighting}
\end{Shaded}

\begin{itemize}
\tightlist
\item
  \textgreater\textgreater{} - extracts one value of variable type and
  stops at a whitespace or mismatch type
\end{itemize}

\begin{Shaded}
\begin{Highlighting}[]
\NormalTok{getline}\OperatorTok{(}\NormalTok{ifstreamObject}\OperatorTok{,}\NormalTok{ strVariable}\OperatorTok{);}
\end{Highlighting}
\end{Shaded}

\begin{itemize}
\tightlist
\item
  recall \texttt{getline()} reads a single line as string into
  strVariable
\end{itemize}

    \begin{tcolorbox}[breakable, size=fbox, boxrule=1pt, pad at break*=1mm,colback=cellbackground, colframe=cellborder]
\prompt{In}{incolor}{5}{\boxspacing}
\begin{Verbatim}[commandchars=\\\{\}]
\PY{c+c1}{// let\PYZsq{}s read couple of words from inputfile.txt}
\PY{n}{string} \PY{n}{word1}\PY{p}{,} \PY{n}{word2}\PY{p}{;}
\end{Verbatim}
\end{tcolorbox}

    \begin{tcolorbox}[breakable, size=fbox, boxrule=1pt, pad at break*=1mm,colback=cellbackground, colframe=cellborder]
\prompt{In}{incolor}{6}{\boxspacing}
\begin{Verbatim}[commandchars=\\\{\}]
\PY{n}{fin} \PY{o}{\PYZgt{}}\PY{o}{\PYZgt{}} \PY{n}{word1} \PY{o}{\PYZgt{}}\PY{o}{\PYZgt{}} \PY{n}{word2}\PY{p}{;}
\end{Verbatim}
\end{tcolorbox}

    \begin{tcolorbox}[breakable, size=fbox, boxrule=1pt, pad at break*=1mm,colback=cellbackground, colframe=cellborder]
\prompt{In}{incolor}{7}{\boxspacing}
\begin{Verbatim}[commandchars=\\\{\}]
\PY{n}{cout} \PY{o}{\PYZlt{}}\PY{o}{\PYZlt{}} \PY{n}{word1} \PY{o}{\PYZlt{}}\PY{o}{\PYZlt{}} \PY{l+s}{\PYZdq{}}\PY{l+s}{ }\PY{l+s}{\PYZdq{}} \PY{o}{\PYZlt{}}\PY{o}{\PYZlt{}} \PY{n}{word2}\PY{p}{;}
\end{Verbatim}
\end{tcolorbox}

    \begin{Verbatim}[commandchars=\\\{\}]
this is
    \end{Verbatim}

    \begin{tcolorbox}[breakable, size=fbox, boxrule=1pt, pad at break*=1mm,colback=cellbackground, colframe=cellborder]
\prompt{In}{incolor}{8}{\boxspacing}
\begin{Verbatim}[commandchars=\\\{\}]
\PY{c+c1}{// let\PYZsq{}s read the rest of the line}
\PY{n}{string} \PY{n}{line}
\end{Verbatim}
\end{tcolorbox}

    \begin{tcolorbox}[breakable, size=fbox, boxrule=1pt, pad at break*=1mm,colback=cellbackground, colframe=cellborder]
\prompt{In}{incolor}{9}{\boxspacing}
\begin{Verbatim}[commandchars=\\\{\}]
\PY{n}{getline}\PY{p}{(}\PY{n}{fin}\PY{p}{,} \PY{n}{line}\PY{p}{)}\PY{p}{;}
\end{Verbatim}
\end{tcolorbox}

    \begin{tcolorbox}[breakable, size=fbox, boxrule=1pt, pad at break*=1mm,colback=cellbackground, colframe=cellborder]
\prompt{In}{incolor}{10}{\boxspacing}
\begin{Verbatim}[commandchars=\\\{\}]
\PY{n}{cout} \PY{o}{\PYZlt{}}\PY{o}{\PYZlt{}} \PY{n}{line}\PY{p}{;}
\end{Verbatim}
\end{tcolorbox}

    \begin{Verbatim}[commandchars=\\\{\}]
 first sentence.
    \end{Verbatim}

    \begin{tcolorbox}[breakable, size=fbox, boxrule=1pt, pad at break*=1mm,colback=cellbackground, colframe=cellborder]
\prompt{In}{incolor}{11}{\boxspacing}
\begin{Verbatim}[commandchars=\\\{\}]
\PY{c+c1}{// let\PYZsq{}s read the next line}
\PY{n}{getline}\PY{p}{(}\PY{n}{fin}\PY{p}{,} \PY{n}{line}\PY{p}{)}\PY{p}{;}
\PY{n}{cout} \PY{o}{\PYZlt{}}\PY{o}{\PYZlt{}} \PY{n}{line}\PY{p}{;}
\end{Verbatim}
\end{tcolorbox}

    \begin{Verbatim}[commandchars=\\\{\}]
this is 2nd sentence
    \end{Verbatim}

    \begin{tcolorbox}[breakable, size=fbox, boxrule=1pt, pad at break*=1mm,colback=cellbackground, colframe=cellborder]
\prompt{In}{incolor}{12}{\boxspacing}
\begin{Verbatim}[commandchars=\\\{\}]
\PY{c+c1}{// let\PYZsq{}s read the next line}
\PY{n}{getline}\PY{p}{(}\PY{n}{fin}\PY{p}{,} \PY{n}{line}\PY{p}{)}\PY{p}{;}
\PY{n}{cout} \PY{o}{\PYZlt{}}\PY{o}{\PYZlt{}} \PY{n}{line}\PY{p}{;}
\end{Verbatim}
\end{tcolorbox}

    \begin{Verbatim}[commandchars=\\\{\}]
some numbers are below
    \end{Verbatim}

    \begin{tcolorbox}[breakable, size=fbox, boxrule=1pt, pad at break*=1mm,colback=cellbackground, colframe=cellborder]
\prompt{In}{incolor}{13}{\boxspacing}
\begin{Verbatim}[commandchars=\\\{\}]
\PY{c+c1}{// let\PYZsq{}s read the 3 numbers}
\PY{k+kt}{int} \PY{n}{nums}\PY{p}{[}\PY{l+m+mi}{3}\PY{p}{]}\PY{p}{;}
\end{Verbatim}
\end{tcolorbox}

    \begin{tcolorbox}[breakable, size=fbox, boxrule=1pt, pad at break*=1mm,colback=cellbackground, colframe=cellborder]
\prompt{In}{incolor}{14}{\boxspacing}
\begin{Verbatim}[commandchars=\\\{\}]
\PY{n}{fin} \PY{o}{\PYZgt{}}\PY{o}{\PYZgt{}} \PY{n}{nums}\PY{p}{[}\PY{l+m+mi}{0}\PY{p}{]} \PY{o}{\PYZgt{}}\PY{o}{\PYZgt{}} \PY{n}{nums}\PY{p}{[}\PY{l+m+mi}{1}\PY{p}{]} \PY{o}{\PYZgt{}}\PY{o}{\PYZgt{}} \PY{n}{nums}\PY{p}{[}\PY{l+m+mi}{2}\PY{p}{]}\PY{p}{;}
\end{Verbatim}
\end{tcolorbox}

    \begin{tcolorbox}[breakable, size=fbox, boxrule=1pt, pad at break*=1mm,colback=cellbackground, colframe=cellborder]
\prompt{In}{incolor}{15}{\boxspacing}
\begin{Verbatim}[commandchars=\\\{\}]
\PY{n}{cout} \PY{o}{\PYZlt{}}\PY{o}{\PYZlt{}} \PY{n}{nums}\PY{p}{[}\PY{l+m+mi}{0}\PY{p}{]} \PY{o}{\PYZlt{}}\PY{o}{\PYZlt{}} \PY{l+s}{\PYZdq{}}\PY{l+s}{ }\PY{l+s}{\PYZdq{}} \PY{o}{\PYZlt{}}\PY{o}{\PYZlt{}} \PY{n}{nums}\PY{p}{[}\PY{l+m+mi}{1}\PY{p}{]} \PY{o}{\PYZlt{}}\PY{o}{\PYZlt{}} \PY{l+s}{\PYZdq{}}\PY{l+s}{ }\PY{l+s}{\PYZdq{}} \PY{o}{\PYZlt{}}\PY{o}{\PYZlt{}} \PY{n}{nums}\PY{p}{[}\PY{l+m+mi}{2}\PY{p}{]}\PY{p}{;}
\PY{c+c1}{// done reading all the contents of the file}
\end{Verbatim}
\end{tcolorbox}

    \begin{Verbatim}[commandchars=\\\{\}]
10 20 30
    \end{Verbatim}

            \begin{tcolorbox}[breakable, size=fbox, boxrule=.5pt, pad at break*=1mm, opacityfill=0]
\prompt{Out}{outcolor}{15}{\boxspacing}
\begin{Verbatim}[commandchars=\\\{\}]
@0x10e65bed0
\end{Verbatim}
\end{tcolorbox}
        
    \hypertarget{close-file}{%
\subsubsection{close file}\label{close-file}}

\begin{itemize}
\tightlist
\item
  use \texttt{close()} method on \texttt{ifstream} objects
\end{itemize}

    \begin{tcolorbox}[breakable, size=fbox, boxrule=1pt, pad at break*=1mm,colback=cellbackground, colframe=cellborder]
\prompt{In}{incolor}{16}{\boxspacing}
\begin{Verbatim}[commandchars=\\\{\}]
\PY{n}{fin}\PY{p}{.}\PY{n}{close}\PY{p}{(}\PY{p}{)}\PY{p}{;}
\end{Verbatim}
\end{tcolorbox}

    \begin{tcolorbox}[breakable, size=fbox, boxrule=1pt, pad at break*=1mm,colback=cellbackground, colframe=cellborder]
\prompt{In}{incolor}{17}{\boxspacing}
\begin{Verbatim}[commandchars=\\\{\}]
\PY{c+c1}{// can check if file is open}
\PY{n}{fin}\PY{p}{.}\PY{n}{is\PYZus{}open}\PY{p}{(}\PY{p}{)}
\end{Verbatim}
\end{tcolorbox}

            \begin{tcolorbox}[breakable, size=fbox, boxrule=.5pt, pad at break*=1mm, opacityfill=0]
\prompt{Out}{outcolor}{17}{\boxspacing}
\begin{Verbatim}[commandchars=\\\{\}]
false
\end{Verbatim}
\end{tcolorbox}
        
    \begin{tcolorbox}[breakable, size=fbox, boxrule=1pt, pad at break*=1mm,colback=cellbackground, colframe=cellborder]
\prompt{In}{incolor}{18}{\boxspacing}
\begin{Verbatim}[commandchars=\\\{\}]
\PY{n}{fin1}\PY{p}{.}\PY{n}{close}\PY{p}{(}\PY{p}{)}\PY{p}{;}
\end{Verbatim}
\end{tcolorbox}

    \hypertarget{ifstream-member-functions}{%
\subsubsection{ifstream member
functions}\label{ifstream-member-functions}}

\begin{itemize}
\tightlist
\item
  there are a bunch of methods available in ifstream objects
\item
  all the methods can be found here with examples:
  https://en.cppreference.com/w/cpp/io/basic\_ifstream
\end{itemize}

    \hypertarget{file-output}{%
\subsection{File output}\label{file-output}}

\begin{itemize}
\tightlist
\item
  steps required to write output data to a file is similar to reading
  data from a file
\item
  3 steps:

  \begin{enumerate}
  \def\labelenumi{\arabic{enumi}.}
  \tightlist
  \item
    Create a new file or open an existing file into append mode
  \item
    Write data to the file
  \item
    Close the file
  \end{enumerate}
\end{itemize}

\hypertarget{create-a-file}{%
\subsubsection{create a file}\label{create-a-file}}

\begin{itemize}
\tightlist
\item
  to write data to a file, first create ofstream object
\item
  create a new file to write data to

  \begin{itemize}
  \tightlist
  \item
    NOTE: if the file exists, it'll truncate/delete contents of the
    existing file
  \end{itemize}
\item
  syntax:
\end{itemize}

\begin{Shaded}
\begin{Highlighting}[]
\CommentTok{// 1. create ofstream object without creating a file}
\NormalTok{ofstream fout}\OperatorTok{;}
\CommentTok{// 2. create/open file with the object}
\NormalTok{fout}\OperatorTok{.}\NormalTok{open}\OperatorTok{(}\StringTok{"output{-}filename"}\OperatorTok{);}

\CommentTok{// create ofstream object and create a given file}
\NormalTok{ofstream fout}\OperatorTok{(}\StringTok{"output{-}filename"}\OperatorTok{);}
\end{Highlighting}
\end{Shaded}

    \begin{tcolorbox}[breakable, size=fbox, boxrule=1pt, pad at break*=1mm,colback=cellbackground, colframe=cellborder]
\prompt{In}{incolor}{19}{\boxspacing}
\begin{Verbatim}[commandchars=\\\{\}]
\PY{c+cp}{\PYZsh{}}\PY{c+cp}{include} \PY{c+cpf}{\PYZlt{}fstream\PYZgt{}}\PY{c+c1}{ // ifstream and ofstream}
\PY{c+cp}{\PYZsh{}}\PY{c+cp}{include} \PY{c+cpf}{\PYZlt{}iostream\PYZgt{}}
\PY{c+cp}{\PYZsh{}}\PY{c+cp}{include} \PY{c+cpf}{\PYZlt{}string\PYZgt{}}
\PY{c+cp}{\PYZsh{}}\PY{c+cp}{include} \PY{c+cpf}{\PYZlt{}iomanip\PYZgt{}}
\PY{c+cp}{\PYZsh{}}\PY{c+cp}{include} \PY{c+cpf}{\PYZlt{}vector\PYZgt{}}
\PY{c+cp}{\PYZsh{}}\PY{c+cp}{include} \PY{c+cpf}{\PYZlt{}algorithm\PYZgt{}}

\PY{k}{using} \PY{k}{namespace} \PY{n+nn}{std}\PY{p}{;}
\end{Verbatim}
\end{tcolorbox}

    \begin{tcolorbox}[breakable, size=fbox, boxrule=1pt, pad at break*=1mm,colback=cellbackground, colframe=cellborder]
\prompt{In}{incolor}{20}{\boxspacing}
\begin{Verbatim}[commandchars=\\\{\}]
\PY{c+c1}{// create output file stream object}
\PY{n}{ofstream} \PY{n}{fout}\PY{p}{;}
\end{Verbatim}
\end{tcolorbox}

    \begin{tcolorbox}[breakable, size=fbox, boxrule=1pt, pad at break*=1mm,colback=cellbackground, colframe=cellborder]
\prompt{In}{incolor}{21}{\boxspacing}
\begin{Verbatim}[commandchars=\\\{\}]
\PY{c+c1}{// create/open file}
\PY{n}{fout}\PY{p}{.}\PY{n}{open}\PY{p}{(}\PY{l+s}{\PYZdq{}}\PY{l+s}{./demos/file\PYZus{}io/outputfile.txt}\PY{l+s}{\PYZdq{}}\PY{p}{)}\PY{p}{;}
\PY{c+c1}{// you should see a new text file created in the same folder where this notebook is}
\end{Verbatim}
\end{tcolorbox}

    \begin{tcolorbox}[breakable, size=fbox, boxrule=1pt, pad at break*=1mm,colback=cellbackground, colframe=cellborder]
\prompt{In}{incolor}{22}{\boxspacing}
\begin{Verbatim}[commandchars=\\\{\}]
\PY{n}{ofstream} \PY{n+nf}{fout1}\PY{p}{(}\PY{l+s}{\PYZdq{}}\PY{l+s}{./demos/file\PYZus{}io/outputfile1.txt}\PY{l+s}{\PYZdq{}}\PY{p}{)}\PY{p}{;}
\PY{c+c1}{// you should see a new text file created in the same folder where this notebook is}
\end{Verbatim}
\end{tcolorbox}

    \hypertarget{write-data}{%
\subsubsection{write data}\label{write-data}}

\begin{itemize}
\tightlist
\item
  writing data to a file is similar to writing data to std output stream
\item
  use \texttt{\textless{}\textless{}} output insertion operator with the
  stream object
\end{itemize}

    \begin{tcolorbox}[breakable, size=fbox, boxrule=1pt, pad at break*=1mm,colback=cellbackground, colframe=cellborder]
\prompt{In}{incolor}{23}{\boxspacing}
\begin{Verbatim}[commandchars=\\\{\}]
\PY{c+c1}{// write data to output file stream}
\PY{n}{fout} \PY{o}{\PYZlt{}}\PY{o}{\PYZlt{}} \PY{l+s}{\PYZdq{}}\PY{l+s}{Hello World!}\PY{l+s}{\PYZdq{}} \PY{o}{\PYZlt{}}\PY{o}{\PYZlt{}} \PY{n}{endl}\PY{p}{;}
\PY{n}{fout1} \PY{o}{\PYZlt{}}\PY{o}{\PYZlt{}} \PY{l+m+mi}{2} \PY{o}{\PYZlt{}}\PY{o}{\PYZlt{}} \PY{l+s}{\PYZdq{}}\PY{l+s}{ + }\PY{l+s}{\PYZdq{}} \PY{o}{\PYZlt{}}\PY{o}{\PYZlt{}} \PY{l+m+mi}{2} \PY{o}{\PYZlt{}}\PY{o}{\PYZlt{}} \PY{l+s}{\PYZdq{}}\PY{l+s}{ = }\PY{l+s}{\PYZdq{}} \PY{o}{\PYZlt{}}\PY{o}{\PYZlt{}} \PY{p}{(}\PY{l+m+mi}{2}\PY{o}{+}\PY{l+m+mi}{2}\PY{p}{)} \PY{o}{\PYZlt{}}\PY{o}{\PYZlt{}} \PY{n}{endl}\PY{p}{;}
\end{Verbatim}
\end{tcolorbox}

    \hypertarget{close-file}{%
\subsubsection{close file}\label{close-file}}

\begin{itemize}
\tightlist
\item
  closing file is important especially that was opened to write
\item
  file remains locked if it's not explictly closed or until the program
  ends
\end{itemize}

    \begin{tcolorbox}[breakable, size=fbox, boxrule=1pt, pad at break*=1mm,colback=cellbackground, colframe=cellborder]
\prompt{In}{incolor}{24}{\boxspacing}
\begin{Verbatim}[commandchars=\\\{\}]
\PY{n}{fout}\PY{p}{.}\PY{n}{close}\PY{p}{(}\PY{p}{)}\PY{p}{;}
\PY{n}{fout1}\PY{p}{.}\PY{n}{close}\PY{p}{(}\PY{p}{)}\PY{p}{;}
\end{Verbatim}
\end{tcolorbox}

    \hypertarget{formatting-file-output}{%
\subsection{Formatting file output}\label{formatting-file-output}}

\begin{itemize}
\tightlist
\item
  \texttt{iomanip} manipulators work excatly the same way for file
  output
\item
  fixed, setw(), setprecision(), left, right, ws, setfill(), etc. all
  can be used to format the contents written to a file
\end{itemize}

    \begin{tcolorbox}[breakable, size=fbox, boxrule=1pt, pad at break*=1mm,colback=cellbackground, colframe=cellborder]
\prompt{In}{incolor}{25}{\boxspacing}
\begin{Verbatim}[commandchars=\\\{\}]
\PY{n}{fout}\PY{p}{.}\PY{n}{open}\PY{p}{(}\PY{l+s}{\PYZdq{}}\PY{l+s}{./demos/file\PYZus{}io/formatted\PYZus{}output.txt}\PY{l+s}{\PYZdq{}}\PY{p}{)}\PY{p}{;}
\end{Verbatim}
\end{tcolorbox}

    \begin{tcolorbox}[breakable, size=fbox, boxrule=1pt, pad at break*=1mm,colback=cellbackground, colframe=cellborder]
\prompt{In}{incolor}{26}{\boxspacing}
\begin{Verbatim}[commandchars=\\\{\}]
\PY{n}{fout} \PY{o}{\PYZlt{}}\PY{o}{\PYZlt{}} \PY{n}{setw}\PY{p}{(}\PY{l+m+mi}{50}\PY{p}{)} \PY{o}{\PYZlt{}}\PY{o}{\PYZlt{}} \PY{n}{setfill}\PY{p}{(}\PY{l+s+sc}{\PYZsq{}}\PY{l+s+sc}{=}\PY{l+s+sc}{\PYZsq{}}\PY{p}{)} \PY{o}{\PYZlt{}}\PY{o}{\PYZlt{}} \PY{l+s}{\PYZdq{}}\PY{l+s}{ }\PY{l+s}{\PYZdq{}} \PY{o}{\PYZlt{}}\PY{o}{\PYZlt{}} \PY{n}{setfill}\PY{p}{(}\PY{l+s+sc}{\PYZsq{}}\PY{l+s+sc}{ }\PY{l+s+sc}{\PYZsq{}}\PY{p}{)} \PY{o}{\PYZlt{}}\PY{o}{\PYZlt{}} \PY{n}{endl}\PY{p}{;}
\end{Verbatim}
\end{tcolorbox}

    \begin{tcolorbox}[breakable, size=fbox, boxrule=1pt, pad at break*=1mm,colback=cellbackground, colframe=cellborder]
\prompt{In}{incolor}{27}{\boxspacing}
\begin{Verbatim}[commandchars=\\\{\}]
\PY{n}{fout} \PY{o}{\PYZlt{}}\PY{o}{\PYZlt{}} \PY{n}{fixed} \PY{o}{\PYZlt{}}\PY{o}{\PYZlt{}} \PY{n}{setprecision}\PY{p}{(}\PY{l+m+mi}{2}\PY{p}{)}\PY{p}{;} 
\PY{n}{fout} \PY{o}{\PYZlt{}}\PY{o}{\PYZlt{}} \PY{n}{setw}\PY{p}{(}\PY{l+m+mi}{25}\PY{p}{)} \PY{o}{\PYZlt{}}\PY{o}{\PYZlt{}} \PY{n}{left} \PY{o}{\PYZlt{}}\PY{o}{\PYZlt{}} \PY{l+s}{\PYZdq{}}\PY{l+s}{Item}\PY{l+s}{\PYZdq{}} \PY{o}{\PYZlt{}}\PY{o}{\PYZlt{}} \PY{n}{setw}\PY{p}{(}\PY{l+m+mi}{25}\PY{p}{)} \PY{o}{\PYZlt{}}\PY{o}{\PYZlt{}} \PY{n}{right} \PY{o}{\PYZlt{}}\PY{o}{\PYZlt{}} \PY{l+s}{\PYZdq{}}\PY{l+s}{Price}\PY{l+s}{\PYZdq{}} \PY{o}{\PYZlt{}}\PY{o}{\PYZlt{}} \PY{n}{endl}\PY{p}{;}
\PY{n}{fout} \PY{o}{\PYZlt{}}\PY{o}{\PYZlt{}} \PY{n}{setw}\PY{p}{(}\PY{l+m+mi}{50}\PY{p}{)} \PY{o}{\PYZlt{}}\PY{o}{\PYZlt{}} \PY{n}{setfill}\PY{p}{(}\PY{l+s+sc}{\PYZsq{}}\PY{l+s+sc}{=}\PY{l+s+sc}{\PYZsq{}}\PY{p}{)} \PY{o}{\PYZlt{}}\PY{o}{\PYZlt{}} \PY{l+s}{\PYZdq{}}\PY{l+s}{ }\PY{l+s}{\PYZdq{}} \PY{o}{\PYZlt{}}\PY{o}{\PYZlt{}} \PY{n}{setfill}\PY{p}{(}\PY{l+s+sc}{\PYZsq{}}\PY{l+s+sc}{ }\PY{l+s+sc}{\PYZsq{}}\PY{p}{)} \PY{o}{\PYZlt{}}\PY{o}{\PYZlt{}} \PY{n}{endl}\PY{p}{;}
\PY{n}{fout} \PY{o}{\PYZlt{}}\PY{o}{\PYZlt{}} \PY{n}{setw}\PY{p}{(}\PY{l+m+mi}{25}\PY{p}{)} \PY{o}{\PYZlt{}}\PY{o}{\PYZlt{}} \PY{n}{left} \PY{o}{\PYZlt{}}\PY{o}{\PYZlt{}} \PY{l+s}{\PYZdq{}}\PY{l+s}{Apple}\PY{l+s}{\PYZdq{}} \PY{o}{\PYZlt{}}\PY{o}{\PYZlt{}} \PY{n}{setw}\PY{p}{(}\PY{l+m+mi}{25}\PY{p}{)} \PY{o}{\PYZlt{}}\PY{o}{\PYZlt{}} \PY{n}{right} \PY{o}{\PYZlt{}}\PY{o}{\PYZlt{}} \PY{l+m+mf}{5.99} \PY{o}{\PYZlt{}}\PY{o}{\PYZlt{}} \PY{n}{endl}\PY{p}{;}
\PY{n}{fout} \PY{o}{\PYZlt{}}\PY{o}{\PYZlt{}} \PY{n}{setw}\PY{p}{(}\PY{l+m+mi}{25}\PY{p}{)} \PY{o}{\PYZlt{}}\PY{o}{\PYZlt{}} \PY{n}{left} \PY{o}{\PYZlt{}}\PY{o}{\PYZlt{}} \PY{l+s}{\PYZdq{}}\PY{l+s}{Carrots}\PY{l+s}{\PYZdq{}} \PY{o}{\PYZlt{}}\PY{o}{\PYZlt{}} \PY{n}{setw}\PY{p}{(}\PY{l+m+mi}{25}\PY{p}{)} \PY{o}{\PYZlt{}}\PY{o}{\PYZlt{}} \PY{n}{right} \PY{o}{\PYZlt{}}\PY{o}{\PYZlt{}} \PY{l+m+mf}{2.55} \PY{o}{\PYZlt{}}\PY{o}{\PYZlt{}} \PY{n}{endl}\PY{p}{;}
\PY{n}{fout} \PY{o}{\PYZlt{}}\PY{o}{\PYZlt{}} \PY{n}{setw}\PY{p}{(}\PY{l+m+mi}{50}\PY{p}{)} \PY{o}{\PYZlt{}}\PY{o}{\PYZlt{}} \PY{n}{setfill}\PY{p}{(}\PY{l+s+sc}{\PYZsq{}}\PY{l+s+sc}{*}\PY{l+s+sc}{\PYZsq{}}\PY{p}{)} \PY{o}{\PYZlt{}}\PY{o}{\PYZlt{}} \PY{l+s}{\PYZdq{}}\PY{l+s}{ }\PY{l+s}{\PYZdq{}} \PY{o}{\PYZlt{}}\PY{o}{\PYZlt{}} \PY{n}{setfill}\PY{p}{(}\PY{l+s+sc}{\PYZsq{}}\PY{l+s+sc}{ }\PY{l+s+sc}{\PYZsq{}}\PY{p}{)} \PY{o}{\PYZlt{}}\PY{o}{\PYZlt{}} \PY{n}{endl}\PY{p}{;}
\end{Verbatim}
\end{tcolorbox}

    \begin{tcolorbox}[breakable, size=fbox, boxrule=1pt, pad at break*=1mm,colback=cellbackground, colframe=cellborder]
\prompt{In}{incolor}{28}{\boxspacing}
\begin{Verbatim}[commandchars=\\\{\}]
\PY{n}{fout}\PY{p}{.}\PY{n}{close}\PY{p}{(}\PY{p}{)}\PY{p}{;}
\PY{c+c1}{// see the contents of formatted\PYZus{}output.txt file}
\end{Verbatim}
\end{tcolorbox}

    \hypertarget{labs}{%
\subsection{Labs}\label{labs}}

\begin{enumerate}
\def\labelenumi{\arabic{enumi}.}
\tightlist
\item
  The following lab demonstrates the usage of file input and output.

  \begin{itemize}
  \tightlist
  \item
    use the partial solution \texttt{fileio.cpp} in
    \href{./labs/fileio/}{labs/fileio} folder
  \item
    use Makefile to compile and debug the file
  \item
    fix all FIXMEs and write \#FIXED\# next to each fixme once fixed
  \end{itemize}
\end{enumerate}

    \hypertarget{exercises}{%
\subsection{Exercises}\label{exercises}}

\begin{enumerate}
\def\labelenumi{\arabic{enumi}.}
\tightlist
\item
  Write a program that computes distance between two points in Cartesian
  coordinates.

  \begin{itemize}
  \tightlist
  \item
    prompt user to enter name of the input file that contains a bunch of
    points

    \begin{itemize}
    \tightlist
    \item
      using a text editor manually create a file with two coordinate
      points (x, y) per line
    \end{itemize}
  \item
    use vector to store points
  \item
    use as many function(s) as possible
  \item
    write at least 3 test cases for each computing functions
  \item
    program continues to run until user wants to quit
  \item
    most of the part is done in Jupyter Notebook demo
  \end{itemize}
\item
  Write a program to compute area and circumference of a circle.

  \begin{itemize}
  \tightlist
  \item
    prompt user to enter name of the input text file that contains a
    bunch of radii of several circles

    \begin{itemize}
    \tightlist
    \item
      using a text editor manually create a file that contains an
      arbitrary number of radii
    \end{itemize}
  \item
    use vector to store data from the input file
  \item
    use as many function(s) as possible
  \item
    write at least 3 test cases for each computing functions
  \item
    program continues to run until user wants to quit
  \end{itemize}
\item
  Write a program to compute area and perimeter of a rectangle.

  \begin{itemize}
  \tightlist
  \item
    prompt user to enter name of the input text file that contains
    lengths and widths of several rectangles

    \begin{itemize}
    \tightlist
    \item
      using a text editor manually create a file with length and width
      of a rectangle per line
    \end{itemize}
  \item
    use as many function(s) as possible
  \item
    write at least 3 test cases for each computing functions
  \item
    program continues to run until user wants to quit
  \end{itemize}
\item
  Write a program to compute area and perimeter of a triangle given 3
  sides.

  \begin{itemize}
  \tightlist
  \item
    prompt user to enter name of the file that contains 3 sides of
    several triangles

    \begin{itemize}
    \tightlist
    \item
      using a text editor manually create a file that contains 3 sides
      of a triangle per line
    \end{itemize}
  \item
    use as many function(s) as possible
  \item
    write at least 3 test cases for each computing functions
  \end{itemize}
\end{enumerate}

\hypertarget{see-a-sample-solution-for-exercise-4-at-demosvectorstriangle}{%
\subsubsection{\texorpdfstring{see a sample solution for exercise 4 at
\url{demos/vectors/triangle/}}{see a sample solution for exercise 4 at demos/vectors/triangle/}}\label{see-a-sample-solution-for-exercise-4-at-demosvectorstriangle}}

\begin{enumerate}
\def\labelenumi{\arabic{enumi}.}
\setcounter{enumi}{4}
\tightlist
\item
  A Grade Book:

  \begin{itemize}
  \tightlist
  \item
    Write a C++ menu-driven program that let's professors keep track of
    students grades with the following requirements:
  \item
    program must use struct to keep track of students grades
  \item
    program prompts user to enter name of the input text file that
    contains students information in the following format

    \begin{itemize}
    \tightlist
    \item
      first name, last name, test1, test2, test3, test4, test5
    \end{itemize}
  \item
    program calculates avearge grade and the letter grade (A-F) based on
    the average grade
  \item
    program sorts the student records based on grade in non-increasing
    order (highest to lowest)
  \item
    program lets user add a new student
  \item
    program lets user update existing student's information
  \item
    program lets user delete existing student
  \item
    program saves the data back into the same input file as a database
  \item
    program creates a cleanly formatted report of students' grades
  \end{itemize}
\item
  Airline Reservation System:

  \begin{itemize}
  \tightlist
  \item
    Write a C++ menu-driven CLI-based program that let's an airline
    company manage airline reservation on a single aircraft they own
    with the following requirements:
  \item
    aircraft has 10 rows with 2 seat on each row
  \item
    program provieds menu option to display all the avaialable seats
  \item
    program provides menu option to let user pick any available seat
  \item
    program provides menu option to creates total sales report
  \item
    program provides menu option to update price of any seat
  \item
    program saves the data into a file
  \end{itemize}
\end{enumerate}

    \hypertarget{kattis-problems}{%
\subsection{Kattis problems}\label{kattis-problems}}

\begin{itemize}
\tightlist
\item
  typically Kattis problems don't require File IO
\item
  almost all Kattis problems require standard IO for data input and
  printing answers
\end{itemize}

    \hypertarget{summary}{%
\subsection{Summary}\label{summary}}

\begin{itemize}
\tightlist
\item
  the notebook covered file streams (input and output)
\item
  learned how to read structured and unstructured data
\item
  write and format output to a output file
\item
  exercises and sample solution(s)
\end{itemize}

    \begin{tcolorbox}[breakable, size=fbox, boxrule=1pt, pad at break*=1mm,colback=cellbackground, colframe=cellborder]
\prompt{In}{incolor}{ }{\boxspacing}
\begin{Verbatim}[commandchars=\\\{\}]

\end{Verbatim}
\end{tcolorbox}


    % Add a bibliography block to the postdoc
    
    
    
\end{document}
